%!TEX program = xelatex
%!TEX root = ./thesis.tex
\chapter{Experiments}
%!TEX program = xelatex
%!TEX root = ./thesis.tex

\section{Experiments on the Basic Task \textit{move0}}\label{sec_exp_move0}

The methods for solving the basic task \textit{move0}} are discussed in this section.

First, we would like to verify the ability to reproduce consistent learning performance discussed in~\cite{henderson2017matters}. We choose the ACKTR~\cite{wu2017scalable} method and test it on the move0 task. As is discussed previously, the move0 task is actually a single-modal environment since the image observation is redundant. Separate neural network parameters are used for the actor network and critic network. The neural networks have the same architecture that consists of two fully connected layers with 64 hidden units after the motion sensor input. The neural network outputs the mean parameter of the policy distribution. The standard deviation of the policy distribution is parameterized by an independent parameter vector. The batch-size is set to 8000, the KL-divergence set to 0.0001 and 20 parallel agents are used to generate experience. The resulting learning performance is shown in Figure~\ref{fig_acktr_reprod}. It appears that the performance of one experiment gets stuck at a score below 3000 while another continues to improve after reaching the score of 4000. 
\begin{figure}[!htbp]
	\includegraphics[width=0.7\textwidth]{images/rec_0403_reprod}
	\centering
	\caption{Inconsistent performance produced by ACKTR agents with the same  parameters but different runs. The vertical axis is the total return averaged over the recent 20 episodes and the horizontal axis is the number of million time-steps}\label{fig_acktr_reprod}
\end{figure}

We would also like to compare the learning performance of the ACKTR method across different learning rates. The experiment result showing the performance of the original ACKTR algorithm with different learning rates is shown in Figure~\ref{fig_acktr_mom_tune}. The result shows that the agents are likely to get stuck at the score of around 3000. 
%The reason might be that while the KL-divergence based trust-region method provides a theoretical guarantee on the monotonic improvement of performance, the agent might converge too early at local minimums and fails to make efficient exploration.
\begin{figure}[!htbp]
	\includegraphics[width=0.7\textwidth]{rec_180521_acktr_mom}
	\centering
	\caption{Performance of ACKTR agents with different KL-divergence constraints. All the agents are trained with batch-size 4000 and 20 parallel agents. The vertical axis is the total return averaged over the recent 200 episodes and the horizontal axis is the number of million time-steps}\label{fig_acktr_mom_tune}
\end{figure}

Apart from that, we would like to verify if the proposed W-KTR method is less prone to local minimums on the task move0. The experiment results comparing the performance of W-KTR agents with different learning rates is shown in Figure~\ref{fig_wass_const_tune}. The result shows that none of the W-KTR agents gets stuck at the local minimum around 3000, and their final performance can reach a total return of around 6000. However, the W-KTR agent appears to have a much slower rate of improvement on the performance before 50 million time-steps.
\begin{figure}[!htbp]
	\includegraphics[width=0.7\textwidth]{rec_180608_wass_const}
	\centering
	\caption{Performance of W-KTR agents with different W2-metric constraints. All the agents are trained with batch-size 4000 and 20 parallel agents. The vertical axis is the total return averaged over the recent 20 episodes and the horizontal axis is the number of million time-steps}\label{fig_wass_const_tune}
\end{figure}

We also test whether applying a decaying Wasserstein constraint in the early phase of the training will improve the learning performance. The experiment of a W-KTR agent with a decaying W-2 constraint from 0.02 to 0.00003 in the first 15 million time-steps is shown in Figure~\ref{fig_wass_decay}. The agent can achieve a more efficient learning performance in the first 50 million time-steps. However, the drawback of this method is that more hyper-parameters need to be tuned.
\begin{figure}[!htbp]
	\includegraphics[width=0.7\textwidth]{rec_180608_wass_decay}
	\centering
	\caption{The performance of a W-KTR agent with a decaying W2-metric from 0.02 to 0.00003 in the first 15 million time-steps. The agent is trained with batch-size 4000 and 20 parallel agents. The vertical axis is the total return averaged over the most recent 20 episodes and the horizontal axis is the number of millions of time-steps}\label{fig_wass_decay}
\end{figure}

The experiment results show that the proposed W-KTR algorithm is able to achieve the same level of final performance as the contemporary state-of-the-art methods. The major drawback is that the algorithm has a long training time due to the slow rate of performance improvement in the early phase. On the contrary, the decaying learning rate feature of KL-divergence based trust region algorithms is able to achieve a good rate of performance improvement in the early phase of training.

The performance of the W-KTR method on several other continuous control environments are shown in Figure \ref{fig:wktr_bench}.
\begin{figure}[!htbp]
	\centering
	\begin{subfigure}[t]{0.4\textwidth}
		\centering
		\includegraphics[width=\textwidth]{rec_wktr_halfcheetah}
		\caption{HalfCheetah}
	\end{subfigure}%
	~ 
	\begin{subfigure}[t]{0.4\textwidth}
		\centering
		\includegraphics[width=\textwidth]{rec_wktr_hopper}
		\caption{Hopper}
	\end{subfigure}
	~ 
	\begin{subfigure}[t]{0.4\textwidth}
		\centering
		\includegraphics[width=\textwidth]{rec_wktr_inverteddoublependulum}
		\caption{InvertedDoublePendulum}
	\end{subfigure}
	~ 
	\begin{subfigure}[t]{0.4\textwidth}
		\centering
		\includegraphics[width=\textwidth]{rec_wktr_invertedpendulum}
		\caption{InvertedPendulum}
	\end{subfigure}
	~ 
	\begin{subfigure}[t]{0.4\textwidth}
		\centering
		\includegraphics[width=\textwidth]{rec_wktr_reacher}
		\caption{Reacher}
	\end{subfigure}
	~ 
	\begin{subfigure}[t]{0.4\textwidth}
		\centering
		\includegraphics[width=\textwidth]{rec_wktr_swimmer}
		\caption{Swimmer}
	\end{subfigure}
	~ 
	\begin{subfigure}[t]{0.4\textwidth}
		\centering
		\includegraphics[width=\textwidth]{rec_wktr_walker}
		\caption{Walker}
	\end{subfigure}
	~ 
	\begin{subfigure}[t]{0.4\textwidth}
		\centering
		\includegraphics[width=\textwidth]{rec_wktr_humanoid}
		\caption{Humanoid}
	\end{subfigure}

	\caption{The performance of W-KTR on several benchmark environments}
	\label{fig:wktr_bench}
\end{figure}

%%!TEX program = xelatex
%!TEX root = ./thesis.tex
\section{Solution to the basic environment: \textit{move0}}
In this section, we compare the proposed W-KTR method with other state-of-art algorithms on the basic \textit{move0} environment.

The performance of W-KTR algorithm is shown in Figure \ref{rec_move0_wktr_tune}:

\begin{figure}[h]
\includegraphics[width=\textwidth]{images/rec_move0_wktr_tune.pdf}
\centering
\caption{Performance of move0 W-KTR agent with diferent hyper-parameter $\delta_W^2$, the x-axis is the number of million timesteps and the y-axis is the total episode reward averaged over the last 200 episodes}
\end{figure}\label{rec_move0_wktr_tune}
It can be shown that the W-KTR agent has a stable final performance, which is not much dependent on the hyper-parameter $\delta_W^2$.

The performance of ACKTR agents is shown in Figure \ref{rec_move0_acktr_tune}:
\begin{figure}[h]
\includegraphics[width=\textwidth]{images/rec_move0_acktr_tune.pdf}
\centering
\caption{Performance of move0 PPO agent with diferent hyper-parameter $\delta_{kl}$, the x-axis is the number of million timesteps and the y-axis is the total episode reward averaged over the last 200 episodes}
\end{figure}\label{rec_move0_acktr_tune}
It can be shown that the ACKTR agent can achieve a faster rate of improvement at the first 10 million steps, but the improvement rate becomes slow as the training proceeds. The agents tend to stuck at sub-optimal policies since the policies converge too fast and cannot effectively adjust once the STD becomes small.

We also run experiments on tuning a PPO agent for the task move0, which is shown in Figre~\ref{rec_move0_ppo_tuning}. The original PPO algorithms has 40 minibatch updates in each batch, which converges too early. We only performa one gradient update over the whole batch in this experiment. The performance of the PPO agent appears to be sensitive to the hyper-parameter, and the agent can't avoid the drop of performance in the late phase of training. The method also needs to trade-off between the early improvement rate and the final performance. In conclusion, the PPO method is not a robust algorithm.
\begin{figure}[h]
\includegraphics[width=\textwidth]{images/rec_move0_ppo_tuning.pdf}
\centering
\caption{Performance of move0 PPO agent with diferent hyper-parameter $learning rate$, the x-axis is the number of million timesteps and the y-axis is the total episode reward averaged over the last 200 episodes}
\end{figure}\label{rec_move0_ppo_tuning}

All the method above use a batch size of 4000 timesteps produced by 20 parallel agents.

\section{Flat solution to a simple Multi-Modality environment: \textit{move1d}}
In the move1d environment, the agent needs to learn from the image input to decide whether to move forward or backward, as well as control according to the states.
We have tried to solve the problem using ACKTR method, presented in Figure~\ref{rec_flatmove1d_acktr_tune}, shows that the agent can easily get stuck at the score of 1000, and cannot improve much even if the agent breaks the bottleneck because the average STD parameter already drops to below 0.04.
\begin{figure}[h]
\includegraphics[width=\textwidth]{images/rec_flatmove1d_acktr_tune.pdf}
\centering
\caption{Performance of move1d ACKTR agent with diferent hyper-parameter $learning rate$, the x-axis is the number of million timesteps and the y-axis is the total episode reward averaged over the last 20 episodes}
\end{figure}\label{rec_flatmove1d_acktr_tune}

We also did a single experiment using the proposed W-KTR method, and is shown in Figure~\ref{rec_flatmove1d_wktr}. The W-KTR agent manages to achieve a fast improvement rate and high final performance after getting stuck at the bottleneck score 1000 for a while.
\begin{figure}[h]
\includegraphics[width=\textwidth]{images/rec_flatmove1d_wktr.pdf}
\centering
\caption{Performance of move1d W-KTR agent with  hyper-parameter $\delta_W^2= 0.0001$, the x-axis is the number of million timesteps and the y-axis is the total episode reward averaged over the last 20 episodes}
\end{figure}\label{rec_flatmove1d_wktr}

This proves that the W-KTR agent is better at handle local minimum and re-adjust the agent's policy even when the policy STD becomes low.



%!TEX program = xelatex
%!TEX root = ./thesis.tex

\section{Experiment on Flat Reinforcement Learning Solutions to Multi-modality Tasks}
\subsection{Conventional Flat reinforcement Learning Methods}\label{sec_multi_modal_flat}
Contemporary end-to-end flat reinforcement learning methods may not be good at solving tasks with multi-modal state space. 
%We suggest that one of the m reason is the image features are much harder to learn than the state. The agent would tend to stuck at a local minimum when it has already learned the state features well, but has not been able to extract any useful information from the image. However, when the agent has finally learned the image features, the policy already converge to a relatively low-entropy distribution, and contemporary reinforcement learning methods cannot perform exploration again at this phase.

Take the task movecont as an example. In this task, a goal direction is sampled uniformly at random from the continuous range of angles: $[0,2\pi)$, at the beginning of each episode. The agent can only obtain the information about the target direction from the image, where there is a spherical object at the corresponding position. It is easy for the agent to learn to balance itself and stay at a stable pose based on the current motion sensor readings, in order to reduce the control cost and contact cost, as well as preventing the game over condition. However, the agent needs to extract the location of the sphere object in the image and move towards that object to solve the task. Learning the image features takes much more data than learning from the motion sensor readings, and the agent may fail to learn anything meaningful from the image while focusing on the motion sensor features. Even if the agent finally manages to extract image features, the policy might already become a low-entropy distribution, and would not be able to make the necessary explorations for the optimal solution. Therefore, an efficient exploration technique is needed for this kind of tasks with multi-modal state space.

We would first like to examine the effectiveness of the conventional entropy regularization method for encouraging exploration of reinforcement learning agents with continuous action space. We use a multi-modal ACKTR agent. The agent uses separate networks for the actor function and the critic function. The image input gets feed into 3 convolutional layers with a filter size of 3 by 3 and a stride of 2, with 16, 8, 4 filters respectively. The output features of the convolutional network are passed to a fully-connected layer of 16 units. The output 16-unit fully connected layer is then concatenated with the motion sensor inputs. Finally, the concatenated feature layer is fed into 2 fully connected layers with 64 units and outputs the mean parameter of the policy. 

The experiment result on the entropy regularization technique is shown in Figure~\ref{rec_ent_reg}. The agents are trained using the ACKTR algorithm minibatch size 2560 and KL-divergence constraint 0.0003. The samples are produced by 32 parallel agents. The results show that the agent fails to learn the reinforcement learning task when the weight of the entropy term is a little bit too large. 

The average Standard Deviation of the policy distribution during training is visualized in Figure~\ref{rec_std_ent_reg}. The figure shows that the entropy regularization term might have too much bias for the agent to perform exploration instead of improving its performance.
\begin{figure}[!htbp]
	\includegraphics[width=\textwidth]{images/rec_180609_ent_reg.pdf}
	\centering
	\caption{The performance of agents with different entropy regularization weights. The horizontal axis is the number of million timesteps and the vertical axis is the total episode reward averaged over the last 32 episodes}\label{rec_ent_reg}
\end{figure}

\begin{figure}[!htbp]
	\includegraphics[width=\textwidth]{images/rec_180609_std_ent_reg.pdf}
	\centering
	\caption{Average standard deviation of the policy of agents in Figure~\ref{rec_ent_reg}. The horizontal axis is the number of million timesteps.}\label{rec_std_ent_reg}
\end{figure}

\subsection{Experiment on Exceptional Advantage Regularization}\label{sec_exp_adv_reg}
The effectiveness of the exceptional advantage regularization method on multi-modality tasks is discussed in this section.
First we demonstrate the general patterns of the distribution of the advantage values on the simple multi-modality task moveg2. The task samples a target direction at the beginning of each episode, and the agent is rewarded for moving in the direction.

The experiment result on the average total reward of a normal ACKTR agent without any regularization is shown in Figure~\ref{rec_stat_moveg2}. The average reward per time-step during training is shown in Figure~\ref{rec_stat_moveg2_meanrt}, and the average standard deviation is shown in Figure~\ref{rec_stat_moveg2_std}.


\begin{figure}[!htbp]
	\includegraphics[width=0.7\textwidth]{images/rec_180528_statlog.pdf}
	\centering
	\caption{The performance of the ACKTR agent on the moveg2 task. The horizontal axis is the number of million timesteps and the vertical axis is the total episode reward averaged over the last 20 episodes}\label{rec_stat_moveg2}
\end{figure}

\begin{figure}[!htbp]
	\includegraphics[width=0.7\textwidth]{images/rec_180528_meanrt_statlog.pdf}
	\centering
	\caption{The average reward per time-step of ACKTR agent on the moveg2 task, the horizontal axis is the number of training batches}\label{rec_stat_moveg2_meanrt}
\end{figure}

\begin{figure}[!htbp]
\label{key}	\includegraphics[width=0.7\textwidth]{images/rec_180528_std_statlog.pdf}
	\centering
	\caption{The average standard deviation of the ACKTR agent's policy on the moveg2 task. The horizontal axis is the number of training batches}\label{rec_stat_moveg2_std}
\end{figure}
The distribution of advantage values at the first epoch (epoch 0) is shown in Figure~\ref{vis_stats_0}. It can be seen that the advantage values are distributed uniformly because the critic model is just initialized. Figure~\ref{vis_stats_3000} shows the distribution at epoch 3000. It can be seen that after the critic model has been trained for some time, the marginal distribution advantage value tend to follow a normal distribution with zero mean.  Figure~\ref{vis_stats_4900} shows the distribution of advantage values at epoch 4900. The advantage values are likely to have a positive bias at some samples with low log-likelihood  when the agent just managed to escape from a local minimum.  

A large number of samples with low likelihood but positive advantage values indicates that it might be beneficial to the agent if the degree of exploration is increased. However, figure~\ref{rec_stat_moveg2_std} on the change of average standard deviation shows that the agent actually keeps decreasing the standard deviations of its policy even in this case. Therefore, the application of an exceptional advantage regularization term, aims to solve this problem.
\begin{figure}[!htbp]
	\centering
	\begin{subfigure}[t]{0.5\textwidth}
		\centering
		\includegraphics[width=\textwidth]{vis_stats_0}
		\caption{Epoch 0}
			\label{vis_stats_0}
	\end{subfigure}%
	~ 
	\begin{subfigure}[t]{0.5\textwidth}
		\centering
		\includegraphics[width=\textwidth]{vis_stats_3000}
		\caption{Epoch 3000}
			\label{vis_stats_3000}
	\end{subfigure}
	~ 
	\begin{subfigure}[t]{0.7\textwidth}
		\centering
		\includegraphics[width=\textwidth]{vis_stats_4900}
		\caption{Epoch 4900}
		\label{vis_stats_4900}
	\end{subfigure}
	\caption{The distribution of advantage values of the ACKTR agent at epoch 0, 3000 and 4900 respectively. The horizontal axis is the log-likelihood value}
\end{figure}
%\begin{figure}[!htbp]
%	\includegraphics[width=\textwidth]{images/vis_stats_0.pdf}
%	\centering
%	\caption{The distribution of advantage values of the ACKTR agent at batch 0 on the moveg2 task, the horizontal axis is the log-likelihood value}
%	\label{vis_stats_0}
%\end{figure}
%
%\begin{figure}[!htbp]
%	\includegraphics[width=\textwidth]{images/vis_stats_3000.pdf}
%	\centering
%	\caption{The distribution of advantage values of the ACKTR agent at batch 300 on the moveg2 task, the horizontal axis is the log-likelihood value}
%	\label{vis_stats_3000}
%\end{figure}
%
%\begin{figure}[!htbp]
%	\includegraphics[width=\textwidth]{images/vis_stats_4900.pdf}
%	\centering
%	\caption{The distribution of advantage values of the ACKTR agent at batch 4900 on the moveg2 task, the horizontal axis is the log-likelihood value}
%	\label{vis_stats_4900}
%\end{figure}

The performance of the exceptional advantage regularization (EAR) technique is tested on the 'movecont' task. We keep the same neural network configuration as in section \ref{sec_multi_modal_flat}, with the modification that the EAR technique is used.
The experiment results on the performance of an ACKTR agent with different exceptional advantage regularization weights are shown in Figure~\ref{rec_adv_reg}. The average standard deviation of the policies is also shown in Figure~\ref{rec_std_adv_reg}. All the agents are trained with batch-size 2560 and KL-divergence constraint 0.0003. The weight-0 agent is the same as the original ACKTR agent without any exploration regularization, and it could only achieve a total reward of around 2000 at the end of training. The agent with exceptional advantage regularization weight 0.04 can improve rapidly in the early phase before 40 million timestep, but gets stuck at around 3500. This shows that the exceptional advantage regularization method might have an adverse effect on the final performance. The agent with an exceptional advantage regularization weight of 0.01 achieves the best final performance. Its average standard deviation shows that the agent manages to increase its policy entropy and re-explore the environment after it has escaped from a local minimum.
\begin{figure}[!htbp]
	\includegraphics[width=\textwidth]{images/rec_180606_adv_reg.pdf}
	\centering
	\caption{The performance of agents with different exceptional advantage regularization weights. The horizontal axis is the number of million time-steps and the vertical axis is the total episode reward averaged over the last 32 episodes}\label{rec_adv_reg}
\end{figure}

\begin{figure}[!htbp]
	\includegraphics[width=\textwidth]{images/rec_180606_std_adv_reg.pdf}
	\centering
	\caption{The average standard deviation of agents with different exceptional advantage regularization weights. The horizontal axis is the number of million time-steps and the vertical axis is the total episode reward averaged over the last 32 episodes}\label{rec_std_adv_reg}
\end{figure}

\subsection{Experiment on Robust Concentric Gaussian Mixture Policy Model}
The effectiveness of robust concentric Gaussian mixture policy agent in exploration of task movecont is tested in this section. We use the same configuration as in section \ref{sec_multi_modal_flat} except that the distribution type is changed to the robust concentric Gaussian mixture policy. We use the empirical estimation of the KL-divergence since there is no closed-form solution in it.

The experiment result on the performance of an ACKTR Gaussian mixture policy agent is shown in Figure~\ref{rec_mix}. The Gaussian mixture policy agents have slower rates on performance improvement compared to pure Gaussian policy agents. However, the agent is still able to achieve a good final performance, with a total reward of around 4000.

\begin{figure}[!htbp]
	\includegraphics[width=\textwidth]{images/rec_180612_mix.pdf}
	\centering
	\caption{The performance of ACKTR agents with different KL-divergence constraints. The horizontal axis is the number of million time-steps and the vertical axis is the total episode reward averaged over the last 32 episodes.}\label{rec_mix}
\end{figure}

%!TEX program = xelatex
%!TEX root = ./thesis.tex

\section{Experiments on the Flexible-scheduling Hierarchical Method}
This section discusses the experiment results on the proposed flexible-scheduling hierarchical method.

Experiments are done on a set of target tasks: dynamicg8, reachcont and reachcontreg. The target task dynamicg8 is a typical problem that has multi-modality state-space and smooth reward signal, and the target task reachcont is a typical problem that has both multi-modality state-space and sparse reward signal. The set of source tasks is  \{move0, move1, \dots, move7\} for all the experiments.

The problem of hierarchical reinforcement learning consists of two parts: the training of actuator agents and the decider agent. The two problems will be discussed in separate sections.

\subsection{Performance of flat reinforcement learning solutions}
We would firstly like to examine the performance of contemporary reinforcement learning methods on the two typical tasks: dynamicg8 and reachcont. We use the same parameter settings as in section~\ref{sec_exp_adv_reg}. We also test whether exceptional advantage regularization is helpful in these two tasks.

The result on the task dynamicg8 is shown in Figure~\ref{rec_basline_dynamicg8}, and the result on the task reachcont is shown in Figure~\ref{rec_basline_reachcont}. It can be seen that agents fail to solve the tasks. When the task logic becomes more complex and reward function become more sparse, the difficulty of learning useful image features increases, and the task becomes infeasible for contemporary flat reinforcement learning algorithms to solve.

\begin{figure}[!htbp]
	\includegraphics[width=\textwidth]{images/rec_180719_baseline_dynamicg8.pdf}
	\centering
	\caption{Performance of flat reinforcement learning method on the task dynamicg8, with different weights on exceptional advantage regularization. The horizontal axis is the number of million time-steps and the vertical axis is the total episode reward averaged over the last 200 episodes}\label{rec_basline_dynamicg8}
\end{figure}

\begin{figure}[!htbp]
	\includegraphics[width=\textwidth]{images/rec_180718_basiline_reachcont.pdf}
	\centering
	\caption{Performance of flat reinforcement learning method on the task reachcont, with different weights on exceptional advantage regularization. The horizontal axis is the number of million time-steps and the vertical axis is the total episode reward averaged over the last 200 episodes}\label{rec_basline_reachcont}
\end{figure}


\subsection{Training the Actuator Agents with Domain Randomization by Cross-sampling Initial States}
As is discussed previously, the learning of the actuator agent of a source task from the source task set is not always the same problem as training a flat reinforcement learning agent for that single task. The initial states generated by actuator agents for other source tasks must be handled, but is usually not encountered in the original source task.

We proposed that the domain randomization by cross-sampling initial states method can handle the problem of novel initial states. The performance of this method is tested in this section. We use the same configuration as in section \ref{sec_exp_move0} for each actuator policy.  

The experiment result on the performance of all the actuator agents is plotted in Figure~\ref{rec_8task_training}. It can be seen that the actuator agents get stuck at sub-optimal performance levels. 

\begin{figure}[!htbp]
	\includegraphics[width=\textwidth]{images/rec_180617_joint8.pdf}
	\centering
	\caption{Performance of actuator agents with domain randomization by cross-sampling initial states, the horizontal axis is the number of million time-steps and the vertical axis is the total episode reward averaged over the last 200 episodes}\label{rec_8task_training}
\end{figure}

Therefore, the proposed synchronous scheduling of actuator learning method aims to prevent this problem. The experiment result of the technique is shown in Figure \ref{rec_sync_training}. In this experiment, the policy training of the actuator agents who outperforms the global lowest-performance by 1000 is paused until all other actuator agents outperform this agent. The result shows that all the actuator agents are able to reach a final performance of around 6000, although the performance of different agents diverge initially. This has verified that the proposed method can successfully prevent any actuator agents getting stuck at sub-optimal performance levels. 

\begin{figure}[!htbp]
	\includegraphics[width=\textwidth]{images/rec_180619_sync.pdf}
	\centering
	\caption{Performance of actuator agents using synchronous scheduling of actuator learning , the horizontal axis is the number of million time-steps and the vertical axis is the total episode reward averaged over the last 200 episodes}\label{rec_sync_training}
\end{figure}

Therefore the learning of source tasks is successfully solved by the proposed technique of domain randomization by cross-sampling initial states with synchronous scheduling of actuator learning method.
\subsection{Training the Decider Agent}
The training of the decider agent consists of two phases: the training of the decision policy and the switcher policy. We train them in two separate phases. In the first phase, the switcher policy is initialized so that it outputs a termination signal whenever the current decision policy has been executed for more than $l_c$ time steps, which is set to 10 in our experiments. The decision policy is trained with the switcher policy being fixed. After the performance of the decision policy converges, the switcher policy is then trained with the decision policy fixed. We train the networks with a PPO algorithm.

\subsubsection{Phase 1: Decision policy training}
The architecture of neural networks is the same as the agent in section \ref{sec_multi_modal_flat} except that the type of distribution is categorical distribution instead of Gaussian distribution.
The experiment result on the performance of the decision policy training on dynamicg8 is shown in Figure~\ref{fig:rec_dynamicg8_decider_subt10}. The result shows that the agent is able to achieve a performance of around 2900. The performance is below the theoretically optimal score of 6000, but the result shows that the agent could at least learn a meaningful policy.

\begin{figure}[!htbp]
\centering
\includegraphics[width=\linewidth]{rec_180630_dynamicg8_exp37}
\caption{Decision policy training performance of the task dynamicg8, the horizontal axis is the number of million time-steps and the vertical axis is the total episode reward averaged over the last 200 episodes}
\label{fig:rec_dynamicg8_decider_subt10}
\end{figure}

The performance of the training of decision policy on task reachcont is shown in Figure~\ref{fig:rec_reachc05_decider_subt10}. The agent achieves an average reward of around 0.75, and is considered having solved the task.
\begin{figure}[!htbp]
\centering
\includegraphics[width=1\linewidth]{rec_0702_reacc05_decider_exp2.pdf}
\caption{Decision policy training performance of the task reachcont, the horizontal axis is the number of million time-steps and the vertical axis is the total episode reward averaged over the last 200 episodes}
\label{fig:rec_reachc05_decider_subt10}
\end{figure}

The performance of the training of decision policy on task reachcontreg is shown in Figure~\ref{rec_reachcontreg}. The task appears to be more difficult than reachcont because the episode length is much longer. The hierarchical reinforcement learning agent still manages to solve the task at the end.
\begin{figure}[!htbp]
	\centering
	\includegraphics[width=1\linewidth]{rec_180719_reachcont_2.pdf}
	\caption{Decision policy training performance of the task reachcontreg, the horizontal axis is the number of million time-steps and the vertical axis is the total episode reward averaged over the last 200 episodes}
	\label{rec_reachcontreg}
\end{figure}
%\begin{figure}[!htbp]
%	\centering
%	\includegraphics[width=1\linewidth]{rec_180719_reachcont_2.pdf}
%	\caption{Decision policy training performance of the task reachcontreg, the horizontal axis is the number of million time-steps and the vertical axis is the total episode reward averaged over the last 200 episodes}
%	\label{rec_constreachreg}
%\end{figure}

The result of the training of the decision policy shows that the hierarchical reinforcement learning agent is able to achieve a reasonable level of performance. The result is as expected because previous works in SMDP have been studied intensively.

\subsubsection{Phase 2: Switcher policy training}
At the beginning of this phase, the switcher policy is re-initialized randomly so that it outputs a termination signal with a probability of around 0.5. This reinitialized switcher policy is different from the switcher policy in phase 1 which outputs a termination signal deterministically after the current actuator policy has been executed for a pre-defined number of steps. 

The performance of the training of the switcher policy on the task dynamicg8 is shown in Figure~\ref{fig:rec_dynamicg8_switcher}. The best performance during this phase is nearly 3000, which is slightly better than the best performance of 2500 in phase 1. However, the performance suddenly drops to a very low value after around 30 million training steps.

The average execution length of the decision policies is shown in Figure~\ref{fig:rec_dynamicg8_avesubt}. The average execution length becomes around $10$ when the performance reaches the highest level at 30 million training steps, compared to around $4.5$ in phase 1. Therefore, the decision frequency of the decision policy is reduced. It can also be seen that the average execution length suddenly increases to the maximum value 500 after 30 million training steps.

The training of switcher policy is beneficial in the sense that the agent can improve its performance and also reduce the decision frequency of the decision policy. A video recording of the agent with best performance can be watched at  the website: [\url{https://youtu.be/ytW7rTpgRAs}].

\begin{figure}[!htbp]
	\centering
	\includegraphics[width=\linewidth]{rec_180707_dynamicg8_switcher_wo_subt}
	\caption{Switcher policy training performance of the task dynamicg8, the horizontal axis is the number of million time-steps and the vertical axis is the total episode reward averaged over the last 200 episodes}
	\label{fig:rec_dynamicg8_switcher}
\end{figure}

\begin{figure}[!htbp]
	\centering
	\includegraphics[width=\linewidth]{rec_180707_dynamicg8_switcher_wo_subt_subpolicy_len}
	\caption{Average execution length of the actions of decision policies of the task dynamicg8, the horizontal axis is the number of million time-steps and the vertical axis is the decision policy's average execution length of the last batch.}
	\label{fig:rec_dynamicg8_avesubt}
\end{figure}

The performance of the reachcont switcher policy is shown in Figure~\ref{fig:rec_reachcont_switcher}. The final performance has been improved from $0.75$ in the decision policy training to around $0.81$. The training of switcher policy appears to beneficial to the performance of the target task.

The average execution length of the decision policies is plotted in Figure~\ref{fig:rec_reachcont_switcher_subt}. The final average value of the execution length appears to be reduced to $1.6$, compared to $5$ in the decision policy training phase.

Therefore, the switcher policy is able to improve the reinforcement learning performance at the expense of increasing the decision frequency of the decision policy.

\begin{figure}[!htbp]
	\centering
	\includegraphics[width=\linewidth]{rec_180711_reachcont_switcher}
	\caption{Switcher policy training performance of the task reachcont, the horizontal axis is the number of million time-steps and the vertical axis is the total episode reward averaged over the last 32 episodes}
	\label{fig:rec_reachcont_switcher}
\end{figure}
\begin{figure}[!htbp]
	\centering
	\includegraphics[width=\linewidth]{rec_180711_reachcont_switcher_subt}
	\caption{Average execution length of the actions of decision policies of the task reachcont, the horizontal axis is the number of million time-steps and the vertical axis is the decision policy's average execution length of the last batch.}
	\label{fig:rec_reachcont_switcher_subt}
\end{figure}

The performance of the reachcontreg switcher policy is shown in Figure~\ref{rec_switcher_reachcontreg} and the average execution length of actuator policies is shown in Figure~\ref{rec_switcher_subt_reachcontreg}. A video recording of the agent can be viewed at the website:  [\url{https://youtu.be/Y0uHEpJ7Cck}]. The best performance, which as 95, surpasses the final performance of 70 in phase 1. The average execution length is 1.0 when the performance reaches the highest level. This means that temporal abstraction is also infeasible for this task. The training also follows a pattern that the performance degrades in the later phase of training when the average execution length begins to increase.
\begin{figure}[!htbp]
	\centering
	\includegraphics[width=\linewidth]{rec_180719_switcher_reachcontreg}
	\caption{Switcher policy training performance of the task reachcontreg, the horizontal axis is the number of million time-steps and the vertical axis is the total episode reward averaged over the last 200 episodes}
	\label{rec_switcher_reachcontreg}
\end{figure}

\begin{figure}[!htbp]
	\centering
	\includegraphics[width=\linewidth]{rec_180719_switcher_subt_reachcontreg}
	\caption{Average execution length of the actions of decision policies of the task reachcontreg, the horizontal axis is the number of million time-steps and the vertical axis is the decision policy's average execution length of the last batch.}
	\label{rec_switcher_subt_reachcontreg}
\end{figure}

One reason that the performance degrades in the late phase of switcher policy training could be due to the regularization term that penalize the termination of actuator policy. The term could have a stronger impact at the late phase of training, where the scale of the advantage term in the reinforcement learning loss decreases.

The above experiment results show that it is feasible to train a switcher policy that controls the scheduling intervals of the decision policy in an end-to-end manner. However, a better penalty term on frequent actuator policy termination needs be designed to prevent the degradation of performance in the late phase of training. 
% \section{Preliminary experiment on the hierarchical agent architecture}
% A preliminary experiment has been done on training the proposed agent in the dynamic2d environment. The termination policy sets $b_i = 4.5$ and isn't trained during policy updates. The figure is shown in Figure \ref{rec_180419_fix_ter}.

% The result shows that the learning of the root-level policy is effective.


% \begin{figure}[h]
% \includegraphics[width=\textwidth]{images/rec_180420_fix_ter.pdf}
% \centering
% \caption{Experiment on dynamic2d on a hierarchical reinforcement learning agent with fixed-length termination policy}
% \end{figure}\label{rec_180419_fix_ter}

% The current guess on the source of stability is that the sub-policy doesn't perform well in the target environment. Figure \ref{rec_180419_fix_ter_len} shows the average episode length of the hierachical agent.

% \begin{figure}[h]
% \includegraphics[width=\textwidth]{images/rec_180420_fix_ter_len.pdf}
% \centering
% \caption{The average episode length of the dynamic2d hierarchical agent}
% \end{figure}\label{rec_180419_fix_ter_len}

% \section{Experiment up to April. 26: on transfer of source policies}
% The results show that the source task policy cannot perform well in dynamic2d, because of different initial condition caused by the switching of moving direction.

% An experiment is developed to fine-tune the source task policies during the learning of the target task. However, the problem is on how to fine-tune a Gaussian policy that has already converged with low STD.

% \section{Ongoing development in transfer of actuator policy}
% \section{On training source tasks}
% I found that using a 1-step proximate policy gradient algorithm with an adaptive KL penalty is the most economic method.

% Previously, the ACKTR agent sometimes stuck at a performance around 3000 as discussed in the group meeting in March. The reason could be due to KL divergence constraint is too small. A small change in the mean vector could lead to a large KL divergence when the STD converges to nearly zero. That prevents the agent's policy from improving.

% It is likely that an alternative metric could be better for trust region methods, like L1-error or JS-divergence.
% \subsection{Training the actuator policy from scratch in the target task}
% An experiment is being run to test if the agent can effectively learn the actuator policies. The actuator agent learns stably. However the rate of improvement is slow due to the simultaneous training of all the source tasks. The current per step mean reward reaches 1.5 after 3 days.
% \subsection{Fine-tuning the actuator policies}
% The fine-tuning of the actuator polices seems to fail in the target problem. The performance usually drops during the fine-tuning period. 
% Deeper investigations are undergoing to fix the problem.
% \subsection{Training a transition policy}
% The development of the transition policy is still undergoing.
% \section{On the hierarchical reinforcement learning}
% I found that the current scheduler agent model is unstable in training. 

% I'm trying to develop a mixed type policy network, where the output is a binary variable and the discrete/continuous distribution. The binary variable indicates whether the current actuator policy should be terminated, and the latter one denotes the policy distribution. The relevant properties such as KL divergence for this kind of distribution should be derived and developed.

% \section{Progress by May. 10}
% Current work still focuses on the investigation of transferring actuator policies.
% The current model fails to reproduce the previous positive experiments on multi-modality tasks. I've done several bug-fixes and parameter tuning, however the problem still remains. The cause is still under investigation, one possible cause is related to the stability of actuator policies in terms of failure rate. If an actuator policy is prune to "game over" behaviour when initialized after another actuator policy, the decider agent may tend to constantly choose a fixed policy. 

% \section{Progress by Mar. 17}
% \subsection{Reproduction of the multi-modality environment performance}
% I've just finished fixing the code, so that the new model achieves a reasonable performance in move1d. The performance is shown in Figure \ref{rec_move1d}.
% \begin{figure}[h]
% \includegraphics[width=\textwidth]{images/rec_180517_move1d_better.pdf}
% \centering
% \caption{Experiment on move1d}
% \end{figure}\label{rec_move1d}

% \subsection{Study of generalized advantage estimation}
% During the development of the hierarchical reinforcement learning model, I developed a modified version of \cite{schulman2015high} for hierarchical reinforcement learning methods. I found a problem that the original method didn't correctly normalize the advantage values $A_t$ according to $\sum_{i=t}^{i=t+T}\lambda^{i}$. That would lead to a advantage values linearly increasing magnitudes with the increased time to termination. However, I found that the correct method doesn't produce a better result according to the experiments up to now.

% \subsection{Policy gradient methods with Wasserstein metric}
% I found that one problem with the current KL-divergence based trust region methods is that agents will find it difficult to learn after the STD of the Gaussian policy approaches zero, due to the sensitive KL-divergence loss. I found that it seems better to replace the KL-divergence metric with Wasserstein metric for Gaussian policies.

% A preliminary experiment is done on the original Ant task and has shown positive results, which is shown in Figure \ref{rec_wass}. The proposed agent is the policy gradient agent with adaptive Wasserstein metric penalty.

% \begin{figure}[h]
% \includegraphics[width=\textwidth]{images/rec_180517_wasserstein.pdf}
% \centering
% \caption{Experiment comparison on Wasserstein metric and KL divergence. agent\_1 is the Wasserstein agent}

% \end{figure}\label{rec_wass}

