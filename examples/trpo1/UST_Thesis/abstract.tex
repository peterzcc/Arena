%!TEX program = xelatex
%!TEX root = ./thesis.tex
\begin{center}
{\Large \thesistitle}\\
\vspace{20mm}
by \thesisauthor\\
\vspace{15mm}
\departmentname\\
\vspace{10mm}
The Hong Kong University of Science and Technology
\end{center}
\vspace{8mm}
\begin{center}
Abstract
\end{center}
One of the most important problems in artificial intelligence is learning to solve control problems without human supervision. Recent advances in deep reinforcement learning methods have achieved significant progress in this domain. Researchers have solved a number of problems including a subset of Atari games~\cite{mnih2015human}, the Go game~\cite{silver2016mastering}, and several simple robot control environments~\cite{duan2016benchmarking}.  However, a general solution to more realistic problems is still missing. Most real-world robot control problems have multi-modal state spaces, which usually consist of both a low-dimensional motion sensor input and a high-dimensional image sensor input. Apart from that, a smooth and informative reward signal is usually unavailable, and the agent is only provided a reward signal that is sparse and discrete.

We study a set of continuous control problems with multi-modal state space. A subset of the problems also have sparse reward functions. We propose several techniques to improve the performance of flat reinforcement learning methods on the multi-modal state-space problems. The proposed techniques include the Wasserstein actor critic trust-region policy optimization method (W-KTR), the exceptional advantage regularization method, and the robust concentric Gaussian mixture policy model. Experiment results show that the proposed techniques, especially the exceptional advantage regularization method, lead to considerable performance improvement. A hierarchical reinforcement learning method, namely the flexible-scheduling hierarchical method, is proposed for the challenging problems with multi-modal state spaces and sparse rewards. Experiment results show that the flexible-scheduling hierarchical method can solve these problems without domain-specific knowledge given a set of pre-defined source tasks.

\par
\noindent


