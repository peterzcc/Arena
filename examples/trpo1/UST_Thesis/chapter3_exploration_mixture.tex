%!TEX program = xelatex
%!TEX root = ./thesis.tex
\section{Efficient Exploration Through Robust Concentric Gaussian Mixture Policy}
Apart from the EAR method for exploration, we propose to use an alternative probability distribution type, namely Robust Concentric Gaussian Mixture, instead of a diagonal Gaussian policy used in previous works.

The proposed policy distribution, namely Robust Concentric Gaussian Mixture (RCGM) Policy, is a mixture of two Gaussian distributions:
\begin{align}
\pi (a|s) = (1-\alpha_{ex})\mathcal{N}(\mu,\Sigma) + \alpha_{ex} \mathcal{N}(\mu,q_{ex}\Sigma)
\end{align}
where the constant $\alpha_{ex}$ is the weight of second distribution. The weight $\alpha_{ex}$ can take a value in the range of $(0,1)$, such as $0.05$, and the constant $q_{ex} \in (1,)$, for example can be $5$, controls the variance of the second deviation. The larger $q_{ex}$ is, the higher the likelihood will be in the tails of the distribution.

The KL-divergence of two RCGM policy does not have a closed form expression even for this special case. However, an empirical estimation of the KL-divergence can still be computed based on the training samples. Apart from that, the Wasserstein-2 distance between two RCGM policy can be given by:
\begin{align}&W_2^2(\pi_{0}(a|\mu_0,\Sigma_0), \pi_{1}(a|\mu_1,\Sigma_1) =  \\ \nonumber
& \ \ \ \ (1-\alpha_{ex})
W_2^2\big(\mathcal{N}(\mu_0,\Sigma_0), \mathcal{N}(\mu_1,\Sigma_1)\big)
+ \alpha_{ex} W_2^2\big(\mathcal{N}(\mu_0,q_{ex}\Sigma_0), \mathcal{N}(\mu_1,q_{ex}\Sigma_1)\big)
\end{align}


Compared to the commonly used Diagonal Gaussian distribution, the RCGM distribution could be much more robust since it has longer tails. Therefore, it is less likely that the agent gets stuck at sub-optimal policies.

