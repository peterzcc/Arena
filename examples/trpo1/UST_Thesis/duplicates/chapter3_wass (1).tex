\section{Wasserstein Actor Critic Kronecker-factored Trust Region Policy Optimization mehod }
As is reported in \cite{henderson2017matters}, the result of the current state-of-art deep reinforcement learning methods for continuous control, including TRPO, ACKTR, PPO and DDPG are hard to be reproduced, due to they're heavily influenced by a variety of factors including random seed, neural network architecture, activation functions in neural network and software implementation. This means that the state-of-art methods are lack of stability in the agent's final performance and robustness to minor change in parameters. That has greatly diminishes the reliability of contemporary deep reinforcement learning methods.

We propose a new policy optimization method, namely Wasserstein Actor Critic Kronecker-factored Trust Region Policy Optimization (W-KTR). We claim that the proposed method outperform other state-of-art methods in terms of agent's final performance, total training time and reproducibility.

The proposed W-KTR method focus on the problem scope of deep reinforcement learning for continuous control, specifically robot control problems. The action of these environments usually stands for mechanics-related physical quantity, such as motor torque and target motor phase. However, the traditional KL-divergence based algorithms are not suitable for these problems. Because the KL-divergence cannot represent the "deviation of policy" of the problem well, because a small perturbation in the mean value of a policy will lead to a large KL-divergence when the variance  is small. 

Therefore, we propose to use another criteria, which is the Wasserstein metric, to measure the deviation of the policy updates.  We reformulate the trust region policy optimization problem as follows:
\begin{equation}
    \begin{aligned}
&    \underset{\theta}{\text{maximize}} 
&& J(\theta) \\
& \text{subject to } 
&& \overline{W_2}(\pi_{\theta_{old}},\pi_\theta) \leq \delta_{W}\end{aligned}
\end{equation}
where $ \overline{W_2}(\pi_{\theta_{old}},\pi_\theta)$ is the average Wasserstein-2 metric between the old policy and the current policy.

The Wasserstein-2 metric is defined by the following equation \cite{villani2003topics}:
\begin{equation}
    W_2(P,Q) = 
    \inf_{\Gamma \in \mathbb{P}(X \sim P, Y \sim Q)}
    \mathbb{E}_{X,Y \sim \Gamma} \left[ \lVert X-Y \rVert_2^2 \right]^{1/2}
\end{equation}
where $\mathbb{P}(X \sim P, Y \sim Q)$ is the set of all joint distribution of $X,Y$ with marginals $P,Q$ respectively.

Specifically, when $P$ and $Q$ are Gaussian distributions with mean $m_1,m_2$ and covariance matrix $\Sigma_1$ and $\Sigma_2$, the squared value of Wasserstein-2 distance $W_2^2(.)$ is defined as the following equation~\cite{chafai} :
\begin{equation}
    W_2^2(P,Q) = \Vert m_1-m_2\Vert_2^2 +\mathrm{Tr}\left(\Sigma_1+\Sigma_2-2(\Sigma_1^{1/2}\Sigma_2\Sigma_1^{1/2})^{1/2}\right)
\end{equation}

The problem is solved by performing gradient updates iteratively following the natural gradient with the Rianmanian metric as $W_2$ metric:

\begin{equation}
    s=H_\theta\left( \overline{W_2^2}(\pi_{old},\pi) \right)^{-1}g := A^{-1}g
\end{equation}

We use the approximation of the Rianmanian metric tensor A:

\begin{equation}
    A \approx \mathbb{E}_{s_t} \left[
    \nabla_\theta W_2^2(\pi_{old},\pi) \left(\nabla_\theta W_2^2(\pi_{old},\pi) \right)^T
    \right]
\end{equation}

The solution of $s$ is then computed using Kronecker-factor approximation technique~\cite{wu2017scalable}.

The policy is then updated following the gradient direction $s$ with ADAM stochastic optimization algorithm~\cite{kingma2014adam}. The step size is adjusted in an adaptive manner according to the resulting $W_2$ distance at each training step.