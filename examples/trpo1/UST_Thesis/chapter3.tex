%!TEX program = xelatex
%!TEX root = ./thesis.tex
\chapter{Methodology}
%!TEX program = xelatex
%!TEX root = ./thesis.tex
\section{Target Environments}\label{sec_env}
\subsection{Summary of environment design}
We propose a set of reinforcement learning tasks that are suitable for studies in our target problem of multi-modality and sparse environments. The agents' physical systems are consistent with the 3D robot control environments in OpenAI Gym~\cite{openaigym}, while the external environments and reward functions of the target tasks are different from the original environments in OpenAI Gym. There are three kinds of 3D robot agents in OpenAI Gym: Ant, Humanoid and Swimmer. The "Ant" agent is a quadruped robot with 8-dimensional action space, the "Humanoid" agent is a humanoid robot with 16-dimensional action space, and the "Swimmer" agent is a worm-like robot with 2-dimensinoal action space.  The three agents are showed in Figure~\ref{fig_agent_ant}, Figure~\ref{fig_agent_humanoid} and Figure~\ref{fig_agent_swimmer}.

We will mainly focus on target environments based on the "Ant" agent, because the agent has the capability of solving a rich variety of high-level tasks, while the agent's primary control tasks are reasonably challenging. 
Compared to the "Ant" agent, the "Swimmer" agent is too simple and its capability of solving more challenging tasks is limited. The "Humanoid" agent's  on the other hand, has a much more unstable physical system compared to "Ant". 
\begin{figure}[H]
	\includegraphics[width=0.5\textwidth]{images/agent_ant.png}
	\centering
	\caption{The "Ant" agent}\label{fig_agent_ant}
\end{figure}

\begin{figure}[H]
	\includegraphics[width=0.5\textwidth]{images/agent_humanoid.png}
	\centering
	\caption{The "Humanoid" agent}\label{fig_agent_humanoid}
\end{figure}
\begin{figure}[H]
	\includegraphics[width=0.5\textwidth]{images/agent_swimmer.png}
	\centering
	\caption{The "Swimmer" agent}\label{fig_agent_swimmer}
\end{figure}
%To provide a consistent comparison with other studies, the development of 
%!TEX program = xelatex
%!TEX root = ./thesis.tex

\subsection{Detailed environment specification}
We provide a detailed description of the experiment environments in this section. All of the environments are based on the "Ant" task~\cite{openaigym}. In the original "Ant" task, the agent receives a 111-dimensional state input and produces a 8-dimensional action output. The agent's state consists of a 13-dimensional vector that contains robots pose information, a 14-dimensional vector that represents its velocity information, and a 84-dimensional vector that contains contact force information. The information about the absolute position of the agent is not provided.

For the proposed environments, the agent receives not only the 111-dimensional observation about the robot's locomotion status, but also a $64\times 64\times 1$ dimensional grayscale image observation. A sample image observation is shown in Figure~\ref{fig_ant_imgobs}.

\begin{figure}[H]
	\includegraphics[width=0.5\textwidth]{images/ant_imgobs.png}
	\centering
	\caption{A sample image observation of the target environments}
\end{figure}\label{fig_ant_imgobs}

A basic task, namely "move0", is similar to the "Ant" environment in OpenAI Gym~\cite{openaigym} except an extra image observation as state. The agent is required to move toward a specific direction $g_0=(1,0)$, and the reward at each time-step is given by:
\begin{align}
r = v_g + 1-c_p-c_c
\end{align}
Where $v_g=v \cdot g_0$ is the "forward reward", which rewards the agent for moving toward the target direction. A sphere object presents in the environments at a constant distance from the robot agent to represent the target direction.  $c_p$ is the control cost, which is the power that the agent is consuming, and $c_c$ is the contact cost, which penalizes the agent for collisions. The episode terminates when the agent enters the unrecoverable state of being upside-down, or if the episode length reaches 1000 time-steps.

Apart from "move0", a set of similar tasks with different target directions are denoted as "move1", "move2", "move3", ..., "move7". This tasks are the same as "move0" except that the goal direction is different. The image observation is actually redundant for all these low-level tasks, because the agent only needs to move toward one specific direction in their corresponding environments.

Apart from these simple task, we also propose several tasks that has more complexity. We propose several multi-modality environments such as "moveg2", "moveg4", "moveg8", "movecont", "dynamicg8", "dynamiccont". These tasks requires the agent to learn not only from the state representation but also the image representation. We also propose sparse multi-modality environments, such as "reachcont". The agent receives sparse reward signals in these environments compared with the previous environments. The target direction or location is represented by a sphere object and can be seen in the image observation.
The set of all the proposed environments are described in details in table \ref{table_ant_envs}.


\begin{table}[!htbp]

\begin{center}
\begin{tabular}{|c|p{3cm}|p{4cm}|p{4cm}|}
\hline
Task name & Goal & Reward  &  Description \\
\hline\hline
move0 & velocity: $g_0=(1,0)$ &$ v_g+1-c_p-c_c$  & move toward a target direction \\
\hline
move1 & velocity: $g_1=(-1,0)$ &$ v_g+1-c_p-c_c$  & move toward a target direction\\
\hline
move2 & velocity: $g_2=(0,1)$ &$ v_g+1-c_p-c_c$  & move toward a target direction \\
\hline
move3 & velocity: $g_3=(0,-1)$ &$ v_g+1-c_p-c_c$  & move toward a target direction \\ 
\hline 
move4 & velocity: $g_4=(\sqrt{2}/2,\sqrt{2}/2)$ &$ v_g+1-c_p-c_c$  & move toward a target direction \\ 
\hline 
move5 & velocity: $g_5=(-\sqrt{2}/2,-\sqrt{2}/2)$ &$ v_g+1-c_p-c_c$  & move toward a target direction \\ 
\hline 
move6 & velocity: $g_6=(\sqrt{2}/2,-\sqrt{2}/2)$ &$ v_g+1-c_p-c_c$  & move toward a target direction \\ 
\hline 
move7 & velocity: $g_7=(-\sqrt{2}/2,\sqrt{2}/2)$ &$ v_g+1-c_p-c_c$  & move toward a target direction \\ 
\hline 
moveg2 & velocity samples from: $\{g_0,g_1\}$ &$ v_g+1-c_p-c_c$  & each episode has a random sampled target direction \\ \hline
moveg4 & velocity samples from: $\{g_0,g_1,g_2,g_3\}$ &$ v_g+1-c_p-c_c$  & each episode has a random sampled target direction \\ \hline
moveg8 & velocity samples from: $\{g_0,g_1, \dots,g_7\}$ &$ v_g+1-c_p-c_c$  & each episode has a random sampled target direction \\ \hline
movecont & velocity samples from a continuous range of all unit directions&$ v_g+1-c_p-c_c$  & each episode has a random sampled target direction \\ \hline
dynamicg8 &  velocity samples from: $\{g_0,g_1, \dots,g_7\}$ &$ v_g-c_p-c_c$  & the target direction is re-sampled with probability 0.005 at each time-step  \\ \hline
dynamiccont & velocity samples from a continuous range of all unit directions &$ v_g-c_p-c_c$  & the target direction is re-sampled with probability 0.005 at each time-step  \\ \hline
reachg4 & position samples from $\{g_0,g_1,g_2,g_3\}$ & $I(\lVert x-g\rVert_2^2<0.5) - 0.01$  & The agent is terminated when reaching a target position\\ \hline
reachcont & position samples from the unit circle & $I(\lVert x-g\rVert_2^2<0.5) - 0.01$  & The episode is terminated when reaching a target position\\ \hline
reachcontreg & position samples from the unit circle & $5I(\lVert x-g\rVert_2^2<0.5) - 0.01$  & A new target is sampled when the agent has reached the target position\\ \hline
\end{tabular}
\end{center}
 \caption{Summary of Ant-based environments}
\end{table}\label{table_ant_envs}




%!TEX program = xelatex
%!TEX root = ./thesis.tex
\section{Wasserstein Actor Critic Kronecker-factored Trust Region Policy Optimization mehod }
As is reported in \cite{henderson2017matters}, the result of the current state-of-art deep reinforcement learning methods for continuous control, including TRPO, ACKTR, PPO and DDPG are hard to be reproduced, due to they're heavily influenced by a variety of factors including random seed, neural network architecture, activation functions in neural network and software implementation. This means that the state-of-art methods are lack of stability in the agent's final performance and robustness to minor change in parameters. That has greatly diminishes the reliability of contemporary deep reinforcement learning methods.

We propose a new policy optimization method, namely Wasserstein Actor Critic Kronecker-factored Trust Region Policy Optimization (W-KTR). We claim that the proposed method outperform other state-of-art methods in terms of agent's final performance, total training time and reproducibility.

The proposed W-KTR method focus on the problem scope of deep reinforcement learning for continuous control, specifically robot control problems. The action of these environments usually stands for mechanics-related physical quantity, such as motor torque and target motor phase. However, the traditional KL-divergence based algorithms are not suitable for these problems. Because the KL-divergence cannot represent the "deviation of policy" of the problem well, because a small perturbation in the mean value of a policy will lead to a large KL-divergence when the variance  is small. 

Therefore, we propose to use another criteria, which is the Wasserstein metric, to measure the deviation of the policy updates.  We reformulate the trust region policy optimization problem as follows:
\begin{equation}
    \begin{aligned}
&    \underset{\theta}{\text{maximize}} 
&& J(\theta) \\
& \text{subject to } 
&& \overline{W_2}(\pi_{\theta_{old}},\pi_\theta) \leq \delta_{W}\end{aligned}
\end{equation}
where $ \overline{W_2}(\pi_{\theta_{old}},\pi_\theta)$ is the average Wasserstein-2 metric between the old policy and the current policy.

The Wasserstein-2 metric is defined by the following equation \cite{villani2003topics}:
\begin{equation}
    W_2(P,Q) = 
    \inf_{\Gamma \in \mathbb{P}(X \sim P, Y \sim Q)}
    \mathbb{E}_{X,Y \sim \Gamma} \left[ \lVert X-Y \rVert_2^2 \right]^{1/2}
\end{equation}
where $\mathbb{P}(X \sim P, Y \sim Q)$ is the set of all joint distribution of $X,Y$ with marginals $P,Q$ respectively.

Specifically, when $P$ and $Q$ are Gaussian distributions with mean $m_1,m_2$ and covariance matrix $\Sigma_1$ and $\Sigma_2$, the squared value of Wasserstein-2 distance $W_2^2(.)$ is defined as the following equation~\cite{chafai} :
\begin{equation}
    W_2^2(P,Q) = \Vert m_1-m_2\Vert_2^2 +\mathrm{Tr}\left(\Sigma_1+\Sigma_2-2(\Sigma_1^{1/2}\Sigma_2\Sigma_1^{1/2})^{1/2}\right)
\end{equation}

The problem is solved by performing gradient updates iteratively following the natural gradient with the Rianmanian metric as $W_2$ metric:

\begin{equation}
    s=H_\theta\left( \overline{W_2^2}(\pi_{old},\pi) \right)^{-1}g := A^{-1}g
\end{equation}

We use the approximation of the Rianmanian metric tensor A:

\begin{equation}
    A \approx \mathbb{E}_{s_t} \left[
    \nabla_\theta W_2^2(\pi_{old},\pi) \left(\nabla_\theta W_2^2(\pi_{old},\pi) \right)^T
    \right]
\end{equation}

The solution of $s$ is then computed using Kronecker-factor approximation technique~\cite{wu2017scalable}.

The policy is then updated following the gradient direction $s$ with ADAM stochastic optimization algorithm~\cite{kingma2014adam}. The step size is adjusted in an adaptive manner according to the resulting $W_2$ distance at each training step.
%!TEX program = xelatex
%!TEX root = ./thesis.tex
\section{Efficient Exploration Through Exceptional Advantage Regularization}
One of the major reasons that multi-modality robot environments are challenging is because the features in different modalities have different complexities.

In the target case, where there are two modality: the locomotion state vector and the image input, the agent is very likely to get stuck at a local minimum after having learned the features of the locomotion state vectors while haven't learned the image features. However, the policy is at a much lower entropy at the phase when the agent starts to learn image features.

As far as we know, the distribution type of the continuous stochastic policy in reinforcement learning is always chosen to be normal distribution with diagonal covariance matrices in the works on policy gradient methods . We have observed a phenomenon in the training process of policy gradient methods that, the variance of the policy is extremely unlikely to be increased, even when the agent has just escaped from a local minimum. Therefore, a regulation method becomes necessary to encourage the exploration of the agent when it finds that its policy has converged too fast at a sub-optimal point.

A conventional regularization method for encouraging exploration is entropy regularization:

\begin{align}
g' = g +\beta_{ent}\nabla_\theta \mathbb{E}[ H(\pi_\theta(a|s)) ]
\end{align}
Where $\beta_{ent}$ is the weight controlling the penalty on low entropies.
The entropy of a normal distribution $\mathcal{N}(\mu,\Sigma)$ is defined as:

\begin{align}
	H(\pi) =  \frac{1}{2} \ln \mathrm{det}(2\pi e \Sigma)
\end{align}

When the policy distribution is a normal distribution with diagonal covariance matrix, the entropy regularization basically introduces a constant bias in the gradient of the logarithm of variance parameters. As a result, the entropy regularization method is usually hard to tune, and leads to degradation on learning performance.

Here we propose a novel method that can efficiently encourage exploration without large penalty on the agent's learning performance. We add a loss, namely Exceptional Advantage Regularization loss (EAR), to the original gradient of the variance parameters:
\begin{align}
g'_\Sigma = g_\Sigma + \beta_{exc} \mathbb{E} \left[
	\max\left(0,I(\hat{A}^{GAE}>0) \hat{A}^{GAE} \nabla_\Sigma \log \pi (a|s)\right)\right]
\end{align}
where $\beta_{exc} $ is the weight controlling the bias on exploration, and $I(.)$ is the indicator function, which returns 1 if the expression is true, otherwise 0.

The EAR loss add weights for the positive gradients of the variance parameters of the samples with positive advantage values. The reason behind the loss is that we want to add more importance to the samples which produce exceptionally positive advantage value, but with low sampling likelihood. Therefore we first only focus on points with positive advantage values. The problem remaining is to identify the points with low likelihood, namely "exceptional" points. We simply choose these points that produce positive gradients any of the variance parameters.

%!TEX program = xelatex
%!TEX root = ./thesis.tex
\section{Efficient Exploration Through Robust Concentric Mixture Gaussian Policy}
Apart from the EAR method for exploration, we propose to use an alternative probability distribution type, namely Concentric Mixture Gaussian, instead of a diagonal Gaussian policy.

The proposed policy distribution, namely Robust Concentric Mixture Gaussian (RCMG) Policy, is a mixture of two Gaussian distributions:
\begin{align}
\pi (a|s) = (1-\alpha_{ex})\mathcal{N}(\mu,\Sigma) + \alpha_{ex} \mathcal{N}(\mu,q_{ex}\Sigma)
\end{align}
where the constant $\alpha_{ex}$ is the weight of second distribution, which for example can be set at $0.05$, and the constant $q_{ex}>1$, for example $q_{ex}=5$,  controls the standard deviation of the second deviation.
%TODO loglikelihood

Although the KL divergence between the two RCMG policy doesn't have a analytical form, the Wasserstein-2 distance between two RCMG policy can be given by:
\begin{align}&W_2(\pi_{0}(a|\mu_0,\Sigma_0), \pi_{1}(a|\mu_1,\Sigma_1) =  \\ \nonumber
& \ \ \ \ (1-\alpha_{ex})
W_2\big(\mathcal{N}(\mu_0,\Sigma_0), \mathcal{N}(\mu_1,\Sigma_1)\big)
+ \alpha_{ex} W_2\big(\mathcal{N}(\mu_0,q_{ex}\Sigma_0), \mathcal{N}(\mu_1,q_{ex}\Sigma_1)\big)
\end{align}

%TODO advantage
The RCMG policy is much more robust than an ordinary Gaussian policy because it has much longer tails.


\section{Hierarchical reinforcement learning architecture}
We propose to solve the reinforcement learning problem by a two-level hierarchical model. 

The hierarchical model consists of a top-level decider agent and a set of bottom-level actuator agents. The actuator agents' policies are trained im the source task environments. 

The decider agent takes an action at every time-step. It may either decide which actuator-policy should be executed, or simply skip and continue current actuator-policy. Therefore, assume there are $n_a$ sub-policies, the action space of the root agent is an $(n_a+1)$-discrete action space.

The observation space of the root agent consists of 2 parts, original statn (motion-sensor observation, image observation) and the meta state. The meta observation state the current sub-policy being executed and the number of time-steps since the last decision has been made.

Empirically, the decider agent is parameterized by two policy networks, $\theta_s$ and $\theta_d$. The network $\theta_s$, namely switcher network, outputs a binary action that decides whether the agent should simply continue using the current acting actuator policy, or switch to another policy based on the current state. The agent

The network $\theta_d$, namely decider network, outputs an $n_a$-discrete action space, that select the acting agent

The selected leaf agent executes the corresponding sub-policy and computes the primary actions the agent should take for the original environment.

 The overalll decision-making process of the decider agent is shown in Algorithm~\ref{hrl_decision_proc}.

\begin{algorithm}
\caption{The decider agent mechanism}\label{hrl_decision_proc}
\begin{algorithmic}%[1]
\Function{deciderAct}{self,$s_t$}
\State $a_{decider} \sim \pi_{decider}(s_t)$
 \If{$a_{decider} \neq 0$}
 \State $self.currentActuator \gets self.allA
ctuators[a_{decider}-1]$
 \EndIf
\State $a_{actuator} \gets self.currentActuator.act(s_t)$
\State \Return $a_{actuator}$
\EndFunction
\end{algorithmic}
\end{algorithm}


\section{Generalized advantage estimation for hierarchical reinforcement learning agents}
We propose a generalized advantage estimation method for decider agents hierarchical reinforcement learning agents. 

Assume that a decider agent makes decisions at time $t_1,t_2,\dots$, then the execution length of the decisions are $l_i = t_{i+1} - t_i, i=1,2,\dots$.

Then the definition of reward of the decider action at $t_i$ is given by:
\begin{align}
\bar{r}_{t_i} \defeq
 \sum_{l=0}^{t_{i+1}-t_i-1}
  \gamma^l r_{t_i+l}
\end{align}

Define the TD residual $\dv_{t_i}$ for $i=0,1, \dots$by:
\begin{align}
\dv_{t_i} & \defeq -V(s_{t_i}) + \bar{r}_{t_i} + \gamma^{t_{i+1}-t_i} V(s_{t_{i+1}})
\end{align}
Then the k-step advantage estimation is given by:
\begin{alignat}{2}
\hata_{t_i}^{(1)} & \defeq   \dv_{{t_i}} 
 &&=-V(s_{t_i}) + \bar{r}_{t_i} + \gamma^{t_{i+1}-t_i} V(s_{t_{i+1}})\\
\hata_{t_i}^{(2)} 
&\defeq \dv_{t_i} + \gamma^{t_{i+1}-t_i} \dv_{t_{i+1}} 
&&= -V(s_{t_i}) +\bar{r}_{t_i} + \gamma^{t_{i+1}-t_i} \bar{r}_{t_{i+1}} + \gamma^{t_{i+2}-t_i} V(t_{i+2}) 
\hata_t^{(3)} &\defeq \dv_{t} + \gamma \dv_{t+1} + \gamma^2 \dv_{t+2} &&= -V(s_t) + r_t + \gamma r_{t+1} + \gamma^2 r_{t+2} + \gamma^3 V(s_{t+3}) \label{a3}
\end{alignat}
\begin{align}
\begin{split}
\hata_{t_i}^{(k)} 
&\defeq \sum_{d=0}^{k-1} 
\gamma^{t_{i+d}-t_i} \dv_{t_{i+d}} \\
&= -V(s_t) 
+\bar{r}_{t_i} + \gamma^{t_{i+1}-t_i} \bar{r}_{t_{i+1}} 
+ \dots 
+ \gamma^{t_{i+k-1}-t_i} \bar{r}_{t_{i+k-1}} 
+ \gamma^{t_{i+k}-t_i} V(s_{t_{i+k}})
\end{split}
\end{align}
We can define the unnormalized generalized advantage estimator as a exponentially-weighted sum of these k-step advantage estimators~\cite{schulman2015high}:
\begin{align}
\hata_{t_i}^{GAE_{unnorm}(\lambda)}
&\defeq  \hata_{t_i}^{(1)} + \lambda^{t_{i+1}-t_i}  \hata_{t_i}^{(2)} + \lambda^{t_{i+2}-t_i} \hata_{t_i}^{(3)} + \dots + \lambda^{t_{i+k-1}-t_i} \hata_{t_i}^{(k)} \nonumber \\
&=  \dv_{t_i} 
+ \lambda^{t_{i+1}-t_i} (\dv_{t_i} + \gamma^{t_{i+1}-t_i} \dv_{t_{i+1}} ) \\
&\ \ \ \ \  \ \ \ \ \ \ +\lambda^{t_{i+2}-t_i} (\dv_t + \gamma^{t_{i+1}-t_i} \dv_{t_{i+1}} + \gamma^{t_{i+2}-t_i} \dv_{t_{i+2}}) + \dots \nonumber \\
&\ \ \ \ \  \ \ \ \ \ \ +\lambda^{t_{i+k-1}-t_i}  \sum_{d=0}^{k-1} \gamma^{t_{i+d}-t_i} \dv_{t_{i+d}}\\
&= (
\dv_{t_i}  \sum_{b=0}^{k-1} \lambda^{t_{i+k-1}-t_i}
+\gamma^{t_{i+1}-t_i} \dv_{t_{i+1}} \sum_{b=1}^k \lambda^{t_{i+b}-t_i} \nonumber \\
&\ \ \ \ \  \ \ \ \ \ \ +\gamma^{t_{i+2}-t_i} \dv_{t_{i+2}} \sum_{b=2}^k \lambda^{t_{i+b}-t_i}
+\dots \\
&\ \ \ \ \  \ \ \ \ \ \ +\gamma^{t_{i+k-1}-t_i} \dv_{t_{i+k-1}} \lambda^{t_{i+k-1}-t_i})
\nonumber \\
&= \sum_{d=0}^{k-1} \dv_{t_i} \gamma^{t_{i+d}-t_i} \sum_{b=0}^{k-1} \lambda^{t_{i+b}-t_i}
\label{eq:gaelam1}
\end{align}
The normalized generalized advantage estimater is then given by:
\begin{align}
\hata_{t_i}^{GAE(\lambda)}
= \frac{\hata_{t_i}^{GAE_{unorm}(\lambda)}}{\sum_{b=0}^{k-1} \lambda^{t_{i+b}-t_i}}
\end{align}
However, the unormalized GAE estimator is usually used in practice instead of the normalized one, with a postprocessing step of batch normalization to adjust the scale of the advantages. This pratical method usually lead to large advantage scales for experience data at the beginning of the episodes and small scales for the experience data near episode ends.

