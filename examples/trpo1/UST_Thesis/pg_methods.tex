%!TEX program = xelatex
%!TEX root = ./thesis.tex
\section{Policy Gradient Methods}
Policy gradient mothod is a class of reinforcement learning methods that solve the problem by optimizing a parametrized policy model $\pi_\theta (a|s), a\in A, s\in S$, so that the expected return:
$$
    J(\theta) = \mathbb{E}\big[\sum_{t=0}^T\gamma^t r_t \big]
$$
is maximized. The constant $\gamma$ is a discount factor and $T$ is the episode length.

Policy gradient methods maximize the expected return iteratively by esitimating the gradient $g := \nabla  \mathbb{E}\big[\sum_{t=0}^T\gamma^t r_t \big]$ , which has the form \cite{schulman2015high}:
\begin{equation} g = \mathbb{E}_{t,s_t,a_t} \big[
\Psi_t \frac{\nabla_\theta \pi_\theta(a_t|s_t) }{\pi_\theta(a_t|s_t)}
\big]
=\mathbb{E}_{t,s_t,a_t} \big[
\Psi_t \nabla_\theta \log \pi_\theta(a_t|s_t) 
\big]
\label{pg_def}\end{equation}


$\Psi_t$ may be one of the following~\cite{schulman2015high}:
\begin{enumerate}
    \item $\sum_{t=0}^{\infty} r_t$ expected total return
    \item $\sum_{t'=t}^{\infty} r_t$ expected return following $a_t$
    \item $\sum_{t'=t}^{\infty} [r_t - b(s_t)]$
    \item $Q_\pi (s_t, a_t)$
    \item $A_\pi (s_t, a_t)$
    \item $ \hat{A}_t^{(1)}:=  \delta_t = r_t + V_\pi (s_{t+1}) - V_\pi (s_{t})$ : 1-step TD residual
    \item $ \hat{A}_t^{(2)} :=  \delta_t + \gamma \delta_{t+1}= r_t +\gamma r_{t+1} + \gamma^2 V_\pi (s_{t+2}) - V_\pi (s_{t})$ : 2-step TD residual
    \item $\hat{A}_t^{(k)} := \sum_{l=0}^{k-1} \gamma^{l} \delta_{t+l} =   r_t + \gamma r_{t+1} + \dots + \gamma^{k-1} r_{t+k-1} + \gamma^{k} V(s_{t+k}) -V(s_t)$ : k-step TD residual
    \item $GAE(\lambda) = \frac{1}{(1+\lambda+\lambda^2+\dots)} \big(\hat{A}_t^{(1)} + \lambda \hat{A}_t^{(2)}+ \lambda^2 \hat{A}_t^{(3)} +\dots \big) 
    \approx \sum_{l=0}^{\infty} (\gamma \lambda)^l \delta_{t+l}$ , where $\lambda \in [0,1]$ : the generalized advantage estimator proposed by \cite{schulman2015high}
\end{enumerate}
Several methods for updating the policy parameters given the policy gradient have been proposed. The following section will discuss some state-of-art works.
\subsection{Trust Region Policy Optimization}
The method proposed by \cite{schulman2015trust}, namely Trust Region Policy Optimization (TRPO) is one of the early works in deep reinforcement learning for continuous control. The method formulates the policy optimization problem as a constraint optimization problem:
\begin{equation}
    \begin{aligned}
&    \underset{\theta}{\text{maximize}} 
&& J(\theta) \\
& \text{subject to } 
&& \overline{D_{KL}}(\pi_{\theta_{old}}\|\pi_\theta) \leq \delta_{KL}\label{trpo_obj}\end{aligned}
\end{equation}
where $\pi_{\theta_{old}}$ is the old policy parameter before performing  parameter update, and $\overline{D_{KL}}$ is the average KL divergence over the samples, $\delta_{KL}$ is the hyper-parameter that control the step size of updates.

The method solves this problem through 4 steps. The first step computes $g$ following the Equation \ref{pg_def}. 
The second step computes the Riemannian metric tensor of the parameter space of the policy model $A = H_{\theta}\left(\overline{D_{KL}}(\pi_{\theta_{old}}\|\pi_\theta)\right)$ , where $H_{\theta}(.)$ denotes the Hessian matrix with respect to $\theta$. 
The third step obtains which is the natural gradient update direction $s=A^{-1}g$ by solving $g=As$ through the conjugate gradient algorithm.
The fourth step computes the maximum step-size $\beta= \sqrt{2\delta_{KL}/s^TAs}$ and performs a line search to ensure the improvement of the objective function.

The method is able to solve some simple robot control tasks in \cite{openaigym} within a few millions of training samples. However, both the computation of the Hessian matrix and the conjugate gradient algorithm are infeasible for a neural network model with nontrivial size.

\subsection{Kronecker-factored Trust Region Policy Gradient}
The work of \cite{wu2017scalable} proposes to reduce the computation time solving the natural gradient by Kronecker-factored approximation method. The method is named Actor-critic  Kronecker-factored Trust Region method (ACKTR).
The Fisher Information Matrix $F=\mathbb{E}[\nabla_\theta L \nabla_\theta L^T]$, where $L=\log\pi(a|s)$ is used here to approximate the Riemannian metric tensor.
The matrix is approximated by:
\begin{equation*}
    F=\mathbb{E}[\nabla_\theta L \nabla_\theta L^T] = \mathbb{E}[aa^T \otimes \nabla_s L \nabla_s L^T ] 
    \approx \mathbb{E}[aa^T] \otimes \mathbb{E}[\nabla_s L \nabla_s L^T ] 
\end{equation*}
where $a$ is the input activation vector corresponding to the neural network weight parameters and $s$ is the corresponding preactivation vector,
The natural gradient is then solved by:
\begin{equation}
    s=A^{-1}g \approx F^{-1}g \approx \left(\mathbb{E}[aa^T] \otimes \mathbb{E}[\nabla_s L \nabla_s L^T ] \right)^{-1}g = \mathbb{E}[aa^T]^{-1} g  \mathbb{E}[\nabla_s L \nabla_s L^T ]^{-1}
\end{equation}
The method further speeds up the optimization by reusing recently computed statistic matrices and also by performing asynchronous computation. The method manages to reduce the computation time so that it is comparable to current first-order gradient decent algorithms.
The method manages to computationally efficiently perform trust region natural policy gradient optimization without sacrificing the reinforcement learning performance. A remaining problem is that the method is still batch based and has a high memory occupation.

\subsection{Proximal Policy Gradient Method}
The author of \cite{schulman2017proximal} proposes a first-order policy optimization method, namely proximal policy gradient with clipping (PPO). The method clips the surrogate objective according to the following equation:
\begin{equation}
    L_{CLIP}(\theta) = \mathbb{E} \left[ \min\big(r_t(\theta) A_t , \mathrm{clip}(r_t(\theta),1-\epsilon,1+\epsilon)A_t)\big) \right]
\end{equation}
where $r_t(\theta) = \pi_\theta / \pi_{\theta_{old}}$ is the likelihood ratio of the current policy over the sample policy, and $\epsilon \in [0,1]$ is a hyperparameter. 

The algorithm performs minibatch stochastic optimization at each epoch of collected agent's experience, for a number of epochs. Therefore it is actually an off-policy method since the policy has already deviated from the sampling policy once updated for the first minibatch in each epoch.
%The method can achieve a relatively fast improvement rate in the early phase (within 1 million training steps) of training in several continuous control environments. However, the method is more difficult to tune compared to trust region methods, and might become increasingly unstable in the late phase of training. The method might also converge too early and stuck at a local minimum point, because updating the parameter too many times on a single batch might lead to overfitting.

The advantage of PPO method is that it has both a relatively low time computational complexity and memory complexity. It can achieve a fast convergence rate in terms of the reinforcement learning performance (total episode reward). It also provides some kind of constraint at each epoch so that the policy doesn't deviate too far from the sampling policy. Although the PPO method have a high convergence rate, there is no evidence that shows the method could achieve a better final performance.
\subsection{Other Methods in Deep Reinforcement Learning for Continuous Control}
Several off-policy methods have also been proposed in the problem domain of deep reinforcement learning for continuous control, including the Deep Deterministic Policy Gradient Method (DDPG)~\cite{lillicrap2015continuous}, Actor-Critic with Experience Replay (ACER)~\cite{wang2016sample}, and Trust-PCL~\cite{nachum2017trust}. These methods are first order off-policy algorithms, and are not in the main focus of our study because no evidence has shown that any off-policy methods can significantly outperform the above mentioned methods in terms of training time, final performance and the stability of performance during training time.
